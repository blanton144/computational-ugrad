\section{Ordinary Differential Equations}

Differential equations have the form:
\begin{equation}
\frac{\dd{x}}{\dd{t}} = f(x, t)
\end{equation}
and can come both in systems (i.e. multiple equations that have to
remain all true) and can contain higher orders in the derivatives. 
In the case above, we will want to solve for $x(t)$, and $x$ is
referred to as the dependent variable, and $t$ is referred to as the
independent variable. 

\noindent {\bf What is the distinction between ordinary and partial
  differential equations?}

\begin{answer}
{\it Ordinary differential equations} have a single independent
variable, whereas {\it partial differential equations} require
multiple independent variables. 

For example, Poisson's equation:
\begin{equation}
\nabla^2 \rho = 4 \pi G \rho(x,y,z)
\end{equation}
involves derivatives in all three coordinates, and they are all
required in general. You can reduce Poisson's equation to an ODE in
spherical symmetry.
\end{answer}

\noindent {\bf What is the distinction between homogeneous and
  nonhomogeneous ODEs?}

\begin{answer}
Homogeneous ODEs have  no terms without the dependent variable.
Nonhomogeneous ODEs have ``driving'' terms, without the dependent
variable.

For example, a nonhomogeneous equation is:
\begin{equation}
\frac{\dd{^2x}}{\dd{t^2}} = f(t)
\end{equation}
\end{answer}

\noindent {\bf What is the distinction between linear and nonlinear
  ODEs?}

\begin{answer}
Linear ODEs are {\it linear} in the dependent variable (but not necessarily
in the independent variable). 

Thus, the following equation is linear:
\begin{equation}
\frac{\dd{x}}{\dd{t}} = t^3 x(t)
\end{equation}
but this equation is {\it nonlinear}:
\begin{equation}
\frac{\dd{x}}{\dd{t}} = t x^3(t)
\end{equation}

While solutions of homogeneous, linear ODEs may be superposed to find
new solutions, the same is not true of nonlinear equations.  Any two
solutions to the top equation $x_1$ and $x_2$ can be linearly combined
into a new equation, but that's not going to be necessarily true of
(nontrivial) solutions of the bottom equation.
\end{answer}

\noindent {\bf What is meant by ``first-order,'' ``second-order,''
  ``third-order,'' etc. ODEs?}

\begin{answer}
This nomenclature indicates the order of the derivatives involved.
Thus, the following equation is first-order:
\begin{equation}
\frac{\dd{x}}{\dd{t}} = w(x, t)
\end{equation}
but this equation is second-order:
\begin{equation}
  \frac{\dd{x}}{\dd{t}}
  + \alpha(x, t) \frac{\dd^2{x}}{\dd{t}^2}
  = w(x, t)
\end{equation}
\end{answer}

With ODEs, you need to set {\it boundary conditions}. For example, in
the simplest case, you might set a set of conditions at $t=0$
(i.e. {\it initial conditions}). For each dependent variable, the
number of independent boundary conditions you need to set are equal to
the order of the problem. 

You can in principle set conditions at multiple $t$, as long as you
have the right number of conditions. In such cases, it is not
necessarily the case that your conditions can be met under your
equations of course. It definitely complicates the finding of a
solution.  

ODEs of any order can be reduced to systems of first-order
equations. Specifically they can be reduced to a form like:
\begin{eqnarray}
\frac{\dd{w_0}}{\dd{t}} &=& f_0(\vec{w}, t) \cr
\frac{\dd{w_1}}{\dd{t}} &=& f_1(\vec{w}, t) \cr
&\ldots& \cr
\frac{\dd{w_{N-1}}}{\dd{t}} &=& f_{N-1}(\vec{w}, t)
\end{eqnarray}
where no terms appear on the right-hand side that are derivatives of
$\vec{w}$ with $t$.  In vector form this appears as:
\begin{equation}
\frac{\dd{\vec{w}}}{\dd{t}} = \vec{f}(\vec{w}, t)
\end{equation}

But do not confuse these vectors with three-dimensional vectors in
configuration space. More often (usually) they involve vectors in
phase space.

For example, take the second-order equation in three-dimensional space:
\begin{equation}
\frac{\dd{^2\vec{x}}}{\dd{t}^2} = - \vec{x} + \vec{f}(t)
\end{equation}
which represents a spring under some time-dependent forcing. This may
be reduced to one dimension as follows:
\begin{eqnarray}
\frac{\dd{\vec{x}}}{\dd{t}} &=& \vec{v} \cr
\frac{\dd{\vec{v}}}{\dd{t}} &=& - \vec{x} + \vec{f}(t)
\end{eqnarray}
which represents six first-order equations in the 6-D phase space
vector $\vec{w} = (\vec{x}, \vec{v})$.

\section{Algorithms for ODEs}

For a general nonlinear ODE, given initial conditions, we will need to
integrate (usually) forward in time numerically. There are a number of
well-developed algorithms for doing this. As in other applications we
have done this semester, the more complex algorithms tend to take the
form of higher-order approximations to the solutions of the problem. 

The simplest ODE integrator that exists is {\it Euler's algorithm}.
Like all the simplest examples in this course, you should probably
never use it, but it is illustrative.

This algorithm takes equal time steps $\Delta t$, starting from
$\vec{w}(t=0)$, just says:
\begin{equation}
\vec{w}(t+\Delta t) = \vec{w}(t) + \Delta t {\dot{\vec{w}}}(t) = 
\vec{w}(t) + \Delta t \vec{f}(\vec{w}, t)
\end{equation}


\noindent {\bf What is the approximation error of this method for each
  step, and for a full integration, as a function of the step size?}

\begin{answer}
  For each step, the real answer can be derived from the Taylor Series:
  \begin{equation}
    \vec{w}(t+\Delta t) = \vec{w}(t) + \Delta t {\dot{\vec{w}}}(t)
    + \frac{\Delta t^2}{2} \ddot{\vec{w}}(t) + \ldots
  \end{equation}
  So for each step the error is given by the third term, which is
  $\mathcal{O}(\Delta t^2)$. The total error is therefore, to leading
  order, the sum of the individual errors in the integral from $t=a$
  to $t=b$:
  \begin{equation}
    E = \sum_{i=0}^{N-1} \frac{\Delta t^2}{2} \ddot{\vec{w}}(t_i)
     = \frac{\Delta t}{2} \sum_{i=0}^{N-1} \Delta t \dot{\vec{f}}(t_i)
     \approx \frac{\Delta t}{2} \int_a^b \dd{t} \dot{\vec{f}}(t)
     = \frac{\Delta t}{2} \left[ f(t=a) - f(t=b)\right].
  \end{equation}
  Therefore, a full integration's approximation error scales as
  $\mathcal{O}(\Delta t)$. In general, if an individual step is
  $\mathcal{O}(\Delta t^N)$ the full integration will be
  $\mathcal{O}(\Delta t^{N-1}$.
\end{answer}

{\it Second-order Runga-Kutta} is simple second-order algorithm
evaluates the function at $t+\Delta t /2$, and then takes the full
step based on the values at the mid-point:
\begin{eqnarray}
\vec{k}_1 &=& \Delta t \vec{f}(\vec{w}_n, t_n) \cr
\vec{k}_2 &=& \Delta t \vec{f}\left(\vec{w}_n + \frac{1}{2} k_1,
t_n + \frac{1}{2}\Delta t\right) \cr
\vec{w}_{n+1} &=& \vec{w}_n + \vec{k}_2
\end{eqnarray}

We can see why this is second-order fairly easily. In the case that we
are integrating just one dependent variable, we are trying to
approximate the integral:
\begin{equation}
\Delta w = \int_0^{\Delta t} \dd{t} f(w(t), t)
\end{equation}
We can Taylor expand $f(w(t), t)$ in $t$:
\begin{equation}
  f(w, t) \approx f\left(\Delta t /2\right) +
  \left(t - \frac{\Delta t}{2}\right) \left[\frac{\dd{f}}{\dd{t}}\right]_{\Delta t /2} + 
  \frac{1}{2} \left(t - \frac{\Delta t}{2}\right)^2
  \left[\frac{\dd{^2f}}{\dd{t}^2}\right]_{\Delta t /2}   + \ldots
\end{equation}
When we insert this into the integral we find the second term vanishes:
\begin{equation}
\Delta w \approx \Delta t f(w(\Delta t /2), \Delta t /2) +
\mathcal{O}(\Delta t^3)
\end{equation}
This assumes that we have an estimate of $f(\Delta t /2)$.  However,
since this is multiplied by $\Delta t$, to remain second-order we only
need this estimate to be good to first-order. Thus, the Euler
algorithm (good to first-order) can be used to estimate the midpoint
and we get the procedure above.

This second-order Runge-Kutta algorithm requires two determinations of
$f$, not one. So it will only be useful when the $\Delta t^2$ term is
sufficiently small (i.e. the function is sufficiently smooth), so that
the decrease in time step is worth it.  

Harder to prove, but more useful generally, is the fourth-order
Runge-Kutta. This is given by evaluating the derivative at the start,
using that to estimate the derivative at the mid-point, then using the
mid-point derivative to estimate the mid-point {\it again}, and and
using those results to estimate a trial end point, before reaching the
final answer.
\begin{eqnarray}
\vec{k}_1 &=& \Delta t \vec{f}(\vec{w}_n, t_n) \cr
\vec{k}_2 &=& \Delta t \vec{f}(\vec{w}_n + \frac{1}{2} \vec{k}_1, t_n +
\frac{1}{2} \Delta t) \cr
\vec{k}_3 &=& \Delta t \vec{f}(\vec{w}_n + \frac{1}{2} \vec{k}_2, t_n +
\frac{1}{2} \Delta t) \cr
\vec{k}_4 &=& \Delta t \vec{f}(\vec{w}_n + \vec{k}_3, t_n + \Delta t) \cr
\vec{w}_{n+1}  &=& \vec{w}_n + \frac{1}{6} \vec{k}_1
+ \frac{1}{3} \vec{k}_2 + \frac{1}{3} \vec{k}_3 + \frac{1}{6} \vec{k}_4
\end{eqnarray}
This, it turns out, has error terms $\mathcal{O}(\Delta t^5)$. 

\section{Step sizes}

In integrating differential equations, setting the step size is
critical. Generally you don't know this in advance. ODE integrators
essentially always use an adaptive step size designed to keep the
integration accuracy within some tolerance.

Specifically, this is typically done by performing the integration in
pairs of steps, and then comparing the answer using two steps to using
one step twice as big. This adds at most 50\% more function
evaluations.

The error in one step can be written as $(\Delta t)^5 \phi$, where
$\phi$ is some constant.  The two answers will therefore be:
\begin{eqnarray}
w(t+ 2\Delta t) &=& w_2 + 2 (\Delta t)^5 \phi + \mathcal{O}(\Delta
t^6) \quad\mathrm{two~steps} 
\cr
w(t+ 2\Delta t) &=& w_1 + (2\Delta t)^5 \phi + \mathcal{O}(\Delta t^6)
\quad\mathrm{one~step} 
\end{eqnarray}
where $\phi$ is a constant of $\mathcal{O}(\Delta t^5)$. The second
equation is clearly true. The first one is a little less obvious,
since surely $\phi$ depends on the value of $\vec{f}$; to understand
it, remember leading order terms of the first $\Delta t$ step already
are fifth order, and $\vec{f}$ is multiplied by $\Delta t$ to get
$\vec{w}$, so the differences are higher order than fifth. 

So then the difference gives you:
\begin{equation}
  \left|w_2 - w_1\right| \approx 30 (\Delta t)^5 \phi
\end{equation}
The single-step (of $\Delta t$) absolute value error can therefore be
estimated as:
\begin{equation}
\epsilon \approx \Delta t^5 \phi \approx \frac{\left|w_2 -
  w_1\right|}{30}
\end{equation}

Therefore, we can take the following approach to keep errors at some
specified target level $\delta$, in {\it per unit time} units. Per
time step $\Delta t$, the corresponding error per step would be
$\epsilon = \delta\Delta t$.

After each pair of steps, calculate $\epsilon$. If $\epsilon < \delta \Delta
t $, continue from $t+ 2\Delta t$ (using $w_2$) but change the
step size to a new one $\Delta t_n$ going forward to satisfy:
\begin{equation}
  \epsilon_n = \delta \Delta t_n = \Delta t_n^5 \phi
  = \frac{\Delta t_n^5}{\Delta t^5} \Delta t^5 \phi
  = \frac{\Delta t_n^5}{\Delta t^5} \frac{\left|w_2 - w_1\right|}{30}
\end{equation}
which means:
\begin{equation}
\Delta t_n^4 = \Delta t^5 \frac{30 \delta}{\left|w_2 - w_1\right|}
\end{equation}
and therefore:
\begin{equation}
\Delta t_n = \Delta t \left(\frac{30 \delta \Delta t}{\left|w_2 -
  w_1\right|}\right)^{1/4}
\end{equation}
It is wise to limit the increase in $\Delta t$ (the right hand factor)
to no more than a factor of two or so.

But if $\epsilon > \delta \Delta t $, then also set $\Delta t_n$
using the same formula, but start over from $t$ (that is, do not
accept the initial measurement using the initial $\Delta t$.

There is a further improvement that can be made. We have an estimate
of the one step error with step-size $\Delta t$ to accuracy
$\mathcal{O}(\Delta t^6)$:
\begin{equation}
  (\Delta t)^5 \phi \approx \frac{w_2 - w_1}{30}
\end{equation}
So you can correct the two-step estimate $w_1$:
\begin{equation}
w(t+ 2\Delta t) = w_1 + 2 \Delta t^5 \phi = w_1 + 
  \frac{w_2 - w_1}{15}
\end{equation}
for a fifth order integration method. This basically comes for free. 

\section{Leapfrog Method}

Sometimes what we want is not just high order accuracy, but we want
certain facts about the original equations to remain true, perhaps
even at the expense of accuracy. For example, many sets of equations
are required to obey conservation laws, such as momentum, energy,
etc. It is also sometimes nice to guarantee that the equations you are
using are time-reversible, just like the actual equations in physics
can be.

The simplest version of a time-reversible integrator is the {\it leapfrog
  method}. It is very similar to the second-order Runge-Kutta method. 
However, instead of:
\begin{eqnarray}
\vec{k}_1 &=& \Delta t \vec{f}(\vec{w}_n, t_n) \cr
\vec{w}_{n+1/2} &=& \vec{w}_n + \frac{1}{2}\vec{k}_1 \cr
\vec{k}_2 &=& \Delta t \vec{f}\left(\vec{w}_{n+1/2}, t_{n+1/2}\right) \cr
\vec{w}_{n+1} &=& \vec{w}_n + \vec{k}_2
\end{eqnarray}
We define:
\begin{eqnarray}
\vec{k}_1 &=& \Delta t \vec{f}(\vec{w}_n, t_n) \cr
\vec{w}_{n+1/2} &=& \vec{w}_{n-1/2} + \vec{k}_1 \cr
\vec{k}_2 &=& \Delta t \vec{f}(\vec{w}_{n+1/2}, t_{n+1/2}) \cr
\vec{w}_{n+1} &=& \vec{w}_{n} + \vec{k}_2
\end{eqnarray}
This means that if you start with some $\vec{w}_0$, you need to take a
first half-step with Euler's method to $\vec{w}_{1/2}$, and then you
can proceed from there. This is still second-order, but it can be
shown that it is time-reversal-symmetric. This means that on average
it will conserve energy. Believe it or not, this is not true of
Runge-Kutta!

\section{Verlet Method}

In leapfrog, all components of $\vec{w}$ are updated at steps $n$ and
$n+1/2$. However, if there are a set of components of $\vec{w}$ whose
update only depends on the intermediate values of the other
components, and vice-versa, you can actually get away with only
updating some of the components at $n$ and the rest at $n+1/2$. There
is a useful and commonly used form of leapfrog that does exactly this,
known as the {\it Verlet method}.

It is typically applied to classical equations of motion. If we write
a one-dimensional equation of motion:
\begin{eqnarray}
{\dot x} &=& v \cr
{\dot v} &=& F(x)
\end{eqnarray}
we can write: 
\begin{eqnarray}
x_1 &=& x_0 + \Delta v_{1/2} \cr
v_{3/2} &=& v_{1/2} + \Delta F(x_1)
\end{eqnarray}
This will be accurate with remainders of order $\Delta^3$ at each time
step. Note that eventually you want $v_1$, so:
\begin{equation}
v_{1} = v_{1/2} + \Delta F(x_1) / 2
\end{equation}

This method preserves time-reversibility. If I have $v_{3/2}$ and
$x_1$, I can integrate backwards with the same equations to get the
original $v_{1/2}$ and $x_0$. It also preserves phase space; a certain
small volume in phase space would deform under the Verlet method in a
way that conserved its volume. Another less generally useful fact, but
still useful, is that in two or more dimensions in a spherically
symmetric potential, this method conserves angular
momentum. Preserving these facets of the physical system under study
can be very useful.

\section{Bulirsch-Stoer Method}

The {\it Bulirsch-Stoer method} is a somewhat differently structured
method. Here we will discuss it in one dimension, solving:
\begin{equation}
\frac{\dd{x}}{\dd{t}} = f(x,t)
\end{equation}

Like other integrators, you proceed step by step. Here (as in the
Newman book) we will call the step $H$ from $x(t)$ to $x(t+H)$. In the
end, we will be doing a long integration with many steps of size
$H$. But within each one, we will be dividing it into intermediate
steps; for each step $H$ we will take 1 step of size $h_1 =H$, 2 steps
of size $h_2 = H / 2$, 3 steps of size $h_3 = H/3$, 4 steps of size
$h_4 = H / 4$, \ldots up to about 8 steps.  Each will give an answer
for the full integration across step $H$, and we will use the series
to extrapolate to a more exact answer.

The Bulirsch-Stoer works on the basis that the integration method for
the intermediate steps have only even-order error terms. That is, that
the leading order error will scale as $\mathcal{O}(h_n^{2})$, and that
if that is cancelled, the next leading order errors will be
$\mathcal{O}(h_n^4)$, etc. If you have some number of steps $n$, then
you can denote as $R_{n,1}$ the estimate with $\mathcal{O}(h_n^{2})$
errors.  You can denote further as $R_{n,m}$ estimates with errors
$\mathcal{O}(h_n^{2m})$.

$R_{n,m}$ will relate to the real answer as:
\begin{equation}
x(t+H) = R_{n,m} + c_m h_n^{2m} + \mathcal{O}(h_n^{2m+2}).
\end{equation}
If you have also calculated the answer $R_{n-1,m}$ with one less step,
you also have:
\begin{equation}
x(t+H) = R_{n-1,m} + c_m h_{n-1}^{2m} + \mathcal{O}(h_{n-1}^{2m+2}).
\end{equation}
Because you know $h_{n-1} = n h_n / (n-1)$, you can determine $c_m$.
If you know $c_m$, then you can construct a $2m+2$-order estimate:
\begin{equation}
  R_{n,m+1} = R_{n,m} +c_mh_n^{2m},
\end{equation}
and in detail you can show this to be:
\begin{equation}
  R_{n,m+1} = R_{n,m} + \frac{R_{n,m} - R_{n-1,
      m}}{\left[n/(n-1)\right]^{2m} -1}
\end{equation}

This equation allows us to build successively higher-order
estimates. One starts with the one-step estimate $R_{1,1}$ with errors
of order $\mathcal{O}(h_1^2)$. Then you make the two-step estimate
$R_{2,1}$ with errors of order $\mathcal{O}(h_2^2)$. These first two
estimates allow you to construct the two-step estimate of order
$\mathcal{O}(h_2^4)$ from the recursion. Then you can similarly use
the three-step estimate $R_{3,1}$ and use the previous results and the
recursion relation to estimate $R_{3,3}$, with errors of order
$\mathcal{O}(h_3^6)$. This is getting better and better!

The error estimate at each step is $c_mh_n^{2m}$. You keep increasing
$n$ until the error on $R_{n,m}$ is less than the target value, or you
reach some limit, usually $n=8$. If you reach the limit, you make $H$
smaller, in a manner similar to that used in the adaptive step size
approach described above.

Now we just need to find a way to take the intermediate steps which
has an error which is an even function of the step size, so it is only
the terms $h^{2m}$ that need concern us. Notice that in the argument
above we were depending on the errors to have that property. This will
be true for any time-reversal-symmetric method. This is because such
methods will have the property that their error per step must have the
property:
\begin{equation}
\epsilon(-h) = - \epsilon(h)
\end{equation}
So any Taylor expansion of the error per step has only odd
terms. Since the full error of an integration will scale as one order
higher, the full error has only even terms.

Leapfrog has this property, except for at the outset of the
integration, where you need to do one half-step of Euler. That makes
it obviously unsuited to the Bulirsch-Stoer method, which is taking a
small number of steps $h$ within $H$, and the error in that half-step
will matter a lot and introduce undesirable odd-order terms.

Bulirsch-Stoer is designed to use the {\it modified midpoint method}
instead, which does not have this problem. Starting at step $0$:
\begin{eqnarray}
\vec{w}_{1/2} &=& \vec{w}_{0} + \frac{1}{2} h \vec{f}(\vec{w}_0, t_0) \cr
\vec{w}_{1} &=& \vec{w}_{0} + h \vec{f}(\vec{w}_{1/2}, t_{1/2}) \cr
\vec{w}_{3/2} &=& \vec{w}_{1/2} + h \vec{f}(\vec{w}_{1}, t_{1}) \cr
\vec{w}_{2} &=& \vec{w}_{1} + h \vec{f}(\vec{w}_{3/2}, t_{3/2}) \cr
\ldots
\end{eqnarray}
There are two final points $w_{n-1/2}$ and $w_{n}$. We can also extrapolate
$w_{n-1/2}$ to a separate estimate of $w_n$ as follows:
\begin{equation}
w_n' = w_{n-1/2} + \frac{1}{2} h \vec{f}(\vec{w}_{n}, t_{n})
\end{equation}
We can average the two:
\begin{equation}
w_{n, {\rm final}} = \frac{1}{2}\left(w_n + w_n'\right).
\end{equation}
It can be shown that this estimate has the desired property that it
only has even powers of $h$, by using the last step to cancel the odd
powers of $h$ introduced in the first step.

\section{Stiff Equations and Implicit Methods}

For equations with more than one solution (i.e. any problem with more
than one first-order equation) the solutions can have very different
time scales of evolution. For example, there can be a very quickly
evolving solution, sometimes just decaying away. Following this
solution will force your integrator to make many, many steps to avoid
large errors developing, yet won't actually be important to the
dynamics. 

Here's an example, of an autocatalytic chemical reaction:
\begin{eqnarray}
{\dot x} &=& - 0.04 x + 10^4 y z \cr
{\dot y} &=& 0.04 x - 10^4 y z - 3\times 10^7 y^2 \cr
{\dot z} &=& 3\times 10^7 y^2 
\end{eqnarray}
This is due to Robertson (1966). As far as I know, it is not actually
a real set of equations, but there are similar real cases. If you
start this with $x=1$ and $y=z=0$, $x$ and $z$ complete dominate, with
$y$ playing a very minor role; but $y$ has a transient at the very
beginning that is difficult to follow.

A simpler, even more toy version are the equations:
\begin{eqnarray}
{\dot x} &=& 998 x + 1998 y \cr
{\dot y} &=& -999 x - 1999 y 
\end{eqnarray}
where $x(t=0)=1$, $y(t=0)=0$.  You can see that these equations are
very nearly degenerate. If you convert:
\begin{eqnarray}
  x &=& 2u -v \cr
  y &=& -u + v
\end{eqnarray}
then you can find:
\begin{eqnarray}
   x &=& 2 e^{-t} - e^{-1000x} \cr
   y &=& - e^{-t} + e^{-1000x} \cr
\end{eqnarray}

What goes wrong in trying to integrate this? Go back to the single
equation case:
\begin{equation}
 {\dot y} = - c y
\end{equation}
i.e. with the solution $y=e^{-cy}$.

For simplicity let's consider the Euler approach to this problem:
\begin{equation}
y_{n+1} = y_n + h {\dot y}_n = (1-ch) y_n
\end{equation}
If $h>2/c$, then the amplitude of the factor is $>1$, and then this
grows without bound. So for too large a step, the system isn't just
inaccurate, it is {\it unstable}. You can take a higher order method,
but the same principle will apply that you can't take long time steps
relative to $1/c$. 


