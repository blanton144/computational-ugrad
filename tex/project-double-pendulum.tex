\documentclass[11pt, preprint]{aastex}

\usepackage{environ}
\usepackage{xcolor}
\usepackage{hyperref}
\usepackage{rotating}  
\usepackage{amsmath}

\newcommand{\todo}[1]{{\bf #1}}
\newcommand{\dd}[1]{\ensuremath{{\rm d}#1}}
\newcommand{\mat}[1]{\ensuremath{{\bf #1}}}
\DeclareMathOperator{\sech}{sech}

\newif\ifanswers

\NewEnviron{answer}{\ifanswers\color{blue}\expandafter\BODY\fi}

\newenvironment{ditemize}
{ \begin{list}{}{%
\setlength{\topsep}{0pt}% 
\setlength{\partopsep}{3pt}% 
\setlength{\itemsep}{1pt}\setlength{\parsep}{1pt}% 
\setlength{\itemindent}{0pt}\setlength{\listparindent}{12pt}%
\setlength{\leftmargin}{24pt}\setlength{\rightmargin}{0in}%
\setlength{\labelsep}{6pt}\setlength{\labelwidth}{6pt}%
\renewcommand{\makelabel}{\makebox[\labelwidth][l]{$\bullet$\hspace{\fill}}}}}
{\end{list}}


\begin{document}

\title{\bf Computational Physics Project / Double Pendulum}

This project involves the calculation of the trajectory of a double
pendulum, which is a nonlinear system that can experience chaos. 

You should follow the description of the problem in Exercise 8.15 of
Newman, but we will answer some different questions. In that exercise,
the two seqments of the pendulum are both length $l$ and have equal
masses $m$, and the angles $\theta_1$ and $\theta_2$ are defined so
that the heights of the pendula are $h_1 = - l \cos\theta_1$ and $h_2
= - l (\cos\theta_1 + \cos\theta_2)$.

Newman derives the second-order equations of motion using the
Lagrangian (much easier than using force diagrams):
\begin{eqnarray}
2 \ddot\theta_1 + \ddot\theta_2 \cos(\theta_1 - \theta_2) +
\dot\theta_2^2\sin(\theta_1 - \theta_2) + 2 \frac{g}{l} \sin\theta_1
&=& 0, \cr
\ddot\theta_2 + \ddot\theta_1 \cos(\theta_1 - \theta_2) -
\dot\theta_2^2\sin(\theta_1 - \theta_2) + \frac{g}{l} \sin\theta_2
&=& 0.
\end{eqnarray}

One tricky thing with this problem is that $\theta_1$ and $\theta_2$
are in reality limited between $-\pi$ and $\pi$. Your analysis of the
results will need to account for this fact (e.g. for the Poincar{\'e}
section).

\section{Prep work}

\begin{itemize}
\item Derive the first-order equations of motion on page 400.
\item Derive the expression for the total energy of the system
  (i.e. part (a) of the exercise).
\item What are the four {\it stationary points} of this system and the
  energies associated with them? They are not all stable!
\end{itemize}

\section{Creating and testing the integrator}

\begin{itemize}
\item Using fourth-order Runge-Kutta construct a piece of code that
  will solve this problem (it is okay to use the implementation from
  {\tt scipy}). Have it output the time steps to a file (this will be
  convenient for plotting later).
\item Create a piece of code that will solve this with the leap-frog
  method (again with the ability to output the time steps).
\item Create a piece of code that will create an animation of the
  results.
\item Using the initial conditions of part (b) of the exercise,
  compare the answers of the two integrators for the evolution of
  $\theta_1$ and $\theta_2$.
\item Test the convergence of each integrator as a function of time
  step size.
\item With the same initial conditions, test the conservation of
  energy for each integrator.
\end{itemize}

\section{Exploring the chaos of the system}

The behavior of the system is hard to characterize in a simple
way. The usual method to analyze systems like this is through a
Poincar{\'e} section. To produce such a section, we define a specific
value of one variable. At the times that that variable reaches that
value, we evaluate the other variables of the system; the Poincar{\'e}
section plots all of those values.

\begin{itemize}
\item Consider a Poincar{\'e} section defined by the conditions that
  $\theta_1=0$ and $\dot\theta_1>0$; that is, the top pendulum is
  vertical and moving to the right.  (Note I {\it think}, but am not
  {\it sure}, that the second condition does not matter much) We will
  plot $\theta_2$ and $\dot\theta_2$. Because energy is conserved,
  $\dot\theta_1$ is fully determined for each data point in the plot
  (and its sign is defined because of the way you are constructing the
  section). Write a piece of code to analyze your
  integration output and create such a diagram from some given set of
  initial conditions.
 \item Make the Poincar{\'e} section for several different
   energies. Use a number of different initial conditions of each
   energy. Use energies just above the energies of the lowest three
   stationary points, as these illustrate the different regimes of
   behavior.
\item For each of the three energies, pick a particular fiducial
  starting point along with several ($\sim 10$) nearby starting points
  distributed randomly aroudn it. Examine how the distance between the
  trajectories and the fiducial trajectory changes over time (defined
  using a Euclidean metric in the four-dimensional phase space) for
  each case. For chaotic orbits you should expect the distances to
  grow exponentially; the exponential growth rate is called the
  Lyapunov exponent.
\end{itemize}

\section{Extreme Bonus: Lyapunov estimates}

Section 3 of
\href{https://www.sciencedirect.com/science/article/pii/0167278985900119}{Wolf
  et al (1985)} discusses a method to determine the Lyapunov exponents
from a differential equation.

A very advanced bonus element of this project would be to use that
method to characterize the Lyapunov estimate. {\it Definitely only try
  this if you have everything else done! I'm not sure the grad
  students doing this project will be able to do this part easily
  either, so don't worry if you skip this part.}

\end{document}
