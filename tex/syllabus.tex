\documentclass[11pt, preprint]{aastex}
\usepackage{hyperref} 
\usepackage{rotating}
 
\setlength{\footnotesep}{9.6pt}

\newcounter{thefigs}
\newcommand{\fignum}{\arabic{thefigs}}

\newcounter{thetabs}
\newcommand{\tabnum}{\arabic{thetabs}}

\newcounter{address}

\begin{document}

\title{\bf Computational Physics / PHYS-UA 210}
~
~

\noindent This course teaches computational physics at a level
appropriate for undergraduate physics majors.  Classes meet Tuesday
and Thursday 12:30pm to 1:45pm, in Room 1045 of 726 Broadway.  The
textbook is {\it Computational Physics}, by Mark Newman. I will also
ask you sometimes to look online at the
\href{https://jakevdp.github.io/PythonDataScienceHandbook/}{Python
  Data Science Handbook (PDSH)} by Jake Van Der Plas.

\noindent If you have never programmed in Python before then Chapter 2
of the book will require your special attention. There are also many
online resources for learning the basics of Python. I can recommend
Software Carpentry and also Prof.~David Pine's {\it Introduction to
  Python for science and engineering}, available online through Bobst
Library.

\noindent Prof. Blanton's office is Room 941 of 726 Broadway, and his
email is {\tt blanton@nyu.edu}. You can come to ask questions about
computational physics (or any other subject!) on Tuesdays 11:00am to
12:15pm, or by appointment.

\noindent The teaching assistant is Jatin Abacousnac ({\tt
  ja3067@nyu.edu}) Recitation is Thursday, 5pm--6:15pm, in Room 1067
of 726 Broadway. This time will primarily consist of working on
homework assignments.

\noindent The class will be participatory. Please read the assignments
          {\it before} attending class; you will be expected at
          certain points to follow along with calculations on your
          computer.

\noindent There will be no exams in this course, but there will be a
pretty heavy load of assignments:
\begin{itemize}
\item {\it Weekly homeworks}: You may consult with each other about
  the homeworks, but you must write your own individual code and
  report. These are due on the {\it Fridays} of the indicated weeks
  for PS\#5 and later.
\item {\it Semester project}: Performed in groups of two or three
  students each. I have deadlines for two intermediate drafts of this
  project; the projects are designed such that you will be able to
  complete them in stages over the semester based on material
  previously covered in class. The project culminates in a written
  report and a presentation in December.
\end{itemize}
All material handed in will consist of reports written in \LaTeX\ (the
physics standard typesetting system) and as documented Python code
which the TA and professor will be able to run to produce the data and
plots. You will receive rather specific templates and instructions
about code standards to follow!

\noindent Grades are based on problem sets (65\%), the large project
and presentation (25\%), and class participation (10\%).

\noindent The classes will proceed as follows (subject to revision!).

\baselineskip 0pt
\begin{table}[h!]
\footnotesize
\begin{tabular}{|c|c|c|c|}
\hline
{\it Date} & {\it Topic} & {\it Reading} & {\it Problem Sets} \cr  
\hline 
2023-09-05 (T) & Numbers on computers  & Ch.~1, 2, 3 & \cr
2023-09-07 (R) & Arrays             & PDSH, Ch. 1 \& 2 & \cr
2023-09-12 (T) & Numerics           & Ch.~4 & PS\#1 \cr
2023-09-14 (R) & Random Numbers     & Ch.~10.1--10.2 & \cr
2023-09-19 (T) & Integration        & Ch.~5.1.3 & PS\#2 \cr
2023-09-21 (R) & Integration        & Ch.~5.4--5.6 & Teams Determined \cr
2023-09-26 (T) & Integration        & Ch.~5.7--5.9 & PS\#3\cr
2023-09-28 (R) & \begin{minipage}{7cm}
  \begin{center}
  {\it ``Solving Riemann Problems Numerically''} \\
  {\it (Marcus DuPont guest lecture)}
  \end{center}
  \end{minipage} & --- & \cr
2023-10-03 (T) & Differentiation    & Ch.~5.10--5.11 & PS\#4\cr
2023-10-05 (R) & Linear Algebra     & Ch.~6.1 & \cr
2023-10-10 (T) & {\bf Legislative Day, no class}       & & \cr
2023-10-12 (R) & Linear Algebra     & Ch.~6.1 & \cr
2023-10-17 (T) & Eigensystems       & Ch.~6.2 & \cr
2023-10-19 (R) & Eigensystems       & Ch.~6.2 & PS \#5 \cr
2023-10-24 (T) & Root-finding       & Ch.~6.3 & \cr
2023-10-26 (R) & Minimization       & Ch.~6.3 & Project draft \#1 due \cr
2023-10-31 (T) & Minimization       & Ch.~6.4 & \cr
2023-11-02 (R) & Fourier Analysis   & Ch.~7.1--7.3 & PS\#6 \cr
2023-11-07 (T) & Fourier Analysis   & Ch.~7.4 & \cr
2023-11-09 (R) & Ordinary DEs       & Ch.~8.1 & PS\#7\cr
2023-11-14 (T) & Ordinary DEs       & Ch.~8.2 & \cr
2023-11-16 (R) & Ordinary DEs        & Ch.~8.3 & Project draft \#2 due \cr 
2023-11-21 (T) & Ordinary DEs        & Ch.~8.4--8.5 & \cr
2023-11-23 (R) & {\bf Thanksgiving, no class} & Ch.~8.6 & \cr
2023-11-28 (T) & Partial DEs       & Ch.~8.6 & \cr
2023-11-30 (R) & Partial DEs        & Ch.~9.1--9.2 & PS\#8 \cr
2023-12-05 (T) & Partial DEs        & Ch.~9.3 & \cr
2023-12-07 (R) & Partial DEs        & Ch.~9.3  & Final project due\cr
2023-12-12 (T) & Project presentations & --- & \cr
2023-12-14 (R) & Project presentations & --- & PS\#9 \cr
\hline
\end{tabular}
\end{table}

\end{document}

