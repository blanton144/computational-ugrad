\documentclass[11pt, preprint]{aastex}

\usepackage{environ}
\usepackage{xcolor}
\usepackage{hyperref}
\usepackage{rotating}  
\usepackage{amsmath}

\newcommand{\todo}[1]{{\bf #1}}
\newcommand{\dd}[1]{\ensuremath{{\rm d}#1}}
\newcommand{\mat}[1]{\ensuremath{{\bf #1}}}
\DeclareMathOperator{\sech}{sech}

\newif\ifanswers

\NewEnviron{answer}{\ifanswers\color{blue}\expandafter\BODY\fi}

\newenvironment{ditemize}
{ \begin{list}{}{%
\setlength{\topsep}{0pt}% 
\setlength{\partopsep}{3pt}% 
\setlength{\itemsep}{1pt}\setlength{\parsep}{1pt}% 
\setlength{\itemindent}{0pt}\setlength{\listparindent}{12pt}%
\setlength{\leftmargin}{24pt}\setlength{\rightmargin}{0in}%
\setlength{\labelsep}{6pt}\setlength{\labelwidth}{6pt}%
\renewcommand{\makelabel}{\makebox[\labelwidth][l]{$\bullet$\hspace{\fill}}}}}
{\end{list}}


\begin{document}

\title{\bf Computational Physics Project / LIGO}

This project involves finding a gravitational wave in the LIGO data
set. If you do not know what LIGO is \ldots {\it find out!} This is
probably the most significant physics result of your lifetime. It is
a close race between cosmic acceleration, the Higgs particle, and
this, anyway.

In this project, you will use Fourier techniques to find the merging
black hole signal \texttt{GW150914}.  Write a Python program to read
in the LIGO data, do some simple cleaning, and find the signal.

LIGO has two independent observatories, one in Livingston, LA and the
other in Hanford, WA. Both detectors must see a signal within 10ms of
each other for it to count. A window of data (32 seconds) around the
time of the event is available for download here:
			
\begin{itemize}
\item
  \href{https://losc.ligo.org/s/events/GW150914/H-H1_LOSC_4_V1-1126259446-32.hdf5}{\texttt{https://losc.ligo.org/s/events/GW150914/H-H1\_LOSC\_4\_V1-1126259446-32.hdf5}}
\item
  \href{https://losc.ligo.org/s/events/GW150914/L-L1_LOSC_4_V1-1126259446-32.hdf5}{\texttt{https://losc.ligo.org/s/events/GW150914/L-L1\_LOSC\_4\_V1-1126259446-32.hdf5}}
\end{itemize}

These are \texttt{HDF5} files, a binary file format often used in
science. These files are much smaller and quicker to read/write than
plain text files.  The \texttt{h5py} library (included in Anaconda
and available through \texttt{pip}) interacts with these files.  A
function to do so is:
\begin{verbatim}
def loadLIGOdata(filename):
    f = h5py.File(filename, "r")
    strain = f['strain/Strain'][...]
    t0 = f['strain/Strain'].attrs['Xstart']
    dt = f['strain/Strain'].attrs['Xspacing']
    t = t0 + dt * np.arange(strain.shape[0])
    f.close()
    return t, strain
\end{verbatim}
This returns the time (in seconds) and the strain (which has no
units).  The measurements in this data set are evenly sampled at
$4096$ Hz.

The strain $h(t)$ is what is measured by LIGO. You can think of it as
the fractional change in the length of the 4km interferometer arms:
$\Delta L / L$.

\begin{itemize}
\item Plot the strain as a function of time for both
  detectors. Gravitational waves from astrophysical sources produce a
  maximum strain on Earth of about $10^{-21}$. Can you see a
  gravitational wave in the data?  No.  You cannot.  Most of the
  strain is ``noise'' coming from various physical effects in the
  detector.
\item To find how the noise affects the data, plot the
  \emph{periodogram} $P_{hh} = |h_k|^2$ of the data for each detector
  using an FFT.  The periodogram is a simple (not very optimal)
  estimate of the \emph{power spectrum}, which indicates how much
  power is in each Fourier mode of the data.  The periodogram shows a
  lot of power for $f < 30$ Hz, and many spectral lines.  These lines
  correspond to resonances in the LIGO machinery (the cables
  suspending the mirrors, the 60 Hz electrical frequency, etc). These
  are clearly noise!
\item Filtering out particular modes in Fourier space is not
  difficult, simply multiply the FFT of your data $h_f$ by a
  \emph{transfer function} $H(f)$, where $|H(f)| < 1$, then perform
  the inverse transform to see the filtered data.  Here are two very
  simple filters:
\begin{eqnarray}
	H_{\rm step}(f) &=& \frac{1}{1+(f/f_0)^{2n}} \ , \\
	H_{\rm gauss}(f) &=& 1 - \exp\left(-\frac{(f-f_0)^2}{2\sigma_f^2}\right) \ .
\end{eqnarray}
In the above, $f_0$ is the location of the filter and $n$ or
$\sigma_f$ control the width. Describe in words what each filter
does. Construct some test data, and use the filters on it, that
demonstrate what these filters do.
\item LIGO is most sensitive between $35$ Hz and $350$ Hz. Use
  $H_{step}$ with $n\sim8$ to filter out the modes outside this band.
  Plot the resulting waveform and periodogram.
\item See a signal yet? Use the $H_{\rm gauss}$ filter to remove spectral
  lines from the data as well.  Plot the resulting waveform and
  periodogram.
\item This dataset contains the \emph{first ever} detection of a
  binary black hole merger!  What time does it occur at?  Remember: a
  real signal appears in both data sets with a time delay of no more
  than $10 ms$.
\end{itemize}

\end{document}
