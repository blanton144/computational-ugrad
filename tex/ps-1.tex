\documentclass[11pt, preprint]{aastex}

\usepackage{environ}
\usepackage{xcolor}
\usepackage{hyperref}
\usepackage{rotating}  
\usepackage{amsmath}

\newcommand{\todo}[1]{{\bf #1}}
\newcommand{\dd}[1]{\ensuremath{{\rm d}#1}}
\newcommand{\mat}[1]{\ensuremath{{\bf #1}}}
\DeclareMathOperator{\sech}{sech}

\newif\ifanswers

\NewEnviron{answer}{\ifanswers\color{blue}\expandafter\BODY\fi}

\newenvironment{ditemize}
{ \begin{list}{}{%
\setlength{\topsep}{0pt}% 
\setlength{\partopsep}{3pt}% 
\setlength{\itemsep}{1pt}\setlength{\parsep}{1pt}% 
\setlength{\itemindent}{0pt}\setlength{\listparindent}{12pt}%
\setlength{\leftmargin}{24pt}\setlength{\rightmargin}{0in}%
\setlength{\labelsep}{6pt}\setlength{\labelwidth}{6pt}%
\renewcommand{\makelabel}{\makebox[\labelwidth][l]{$\bullet$\hspace{\fill}}}}}
{\end{list}}


\begin{document}

\title{\bf Computational Physics / PHYS-UA 210 / Problem Set \#1
\\ Due September 12, 2023 }

Submit this homework to the TA as a link to a folder called {\tt ps-1}
in the GitHub repository you will create below.

You {\it must} label all axes of all plots, including giving the {\it
  units}, where applicable!!

\begin{enumerate}

  \item Create a GitHub account. Start a new public repository called
    {\tt phys-ua210}. Clone this repository to wherever you are doing
    your work (e.g. your laptop). Create a {\tt ps-1} folder in it to
    store the problem set results.

  \item Familiarize yourself enough with a {\tt matplotlib} and {\tt
    numpy} in Python to plot a Gaussian with zero mean and a standard
    deviation of 3 over the range [$-$10, $+$10]. Make sure the
    Gaussian is normalized correctly. Create an executable script with
    the file name {\tt plot\_gaussian.py} that that writes a PNG file
    called {\tt gaussian.png}. Under Linux or Mac OS, from a Terminal
    you would type ``{\tt python plot\_gaussian.py}'' to run the
    script. Under Windows, you should be able to type just {\tt
      plot\_gaussian.py}''. Do not import any packages other than {\tt
      matplotlib} or {\tt numpy} to make this plot. Put the script and
    PNG file in the {\tt ps-1} folder and push it to GitHub.

  \item Create a short LaTeX document. You may use code on your laptop
    or you may use Overleaf (at {\tt overleaf.com}). Write down a your
    goals for this course, your background in programming and/or
    numerics, and (to the extent you know them) your plans after your
    degree is finished (grad school? industry?  law school?
    etc.). Write just one paragraph! If you write more than a page,
    you have written way too much. You won't be graded on the content
    of this, just whether you do it!! It will also help me understand
    what you want out of the class. In addition, you should include a
    figure with the PNG figure from part (2) above; include a caption
    for the figure. Put the PDF with the document into the {\tt ps-1}
    folder and push it to GitHub.

\end{enumerate}

\end{document}
